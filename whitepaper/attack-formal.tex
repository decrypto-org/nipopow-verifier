\section{Proof of attack on PoPoW}
\label{sec:attack-full}
We now calculate the exact probability of success of the attack sketched in
Section~\ref{sec:attack}.

The attack is parameterized by  parameters $r, \mu$ which are picked by the
adversary. $\mu$ is the superblock level at which the adversary will produce a
proof longer than the honest proof. The modified attack works as follows:
Without loss of generality, fix block $b$ to be Genesis. The adversary always
mines on the secret chain which forks off from genesis, unless a
\emph{superblock generation event} occurs. If a superblock generation event
occurs, then the adversary pauses mining on the secret chain and attempts a
\emph{block suppression attack} on the honest chain. The adversary devotes
exactly $r$ rounds to this suppression attack; then resumes mining on the secret
chain. We show that, despite this simplification (of fixing $r$) which is
harmful to the adversary, the probability of a successful attack is
non-negligible for certain values of the protocol parameters
\footnote{The attack could be further optimized, but we simplify it for
exposition.}.

The adversary monitors the network for superblocks. Whenever an honest party
diffuses an honestly-generated $\mu$-superblock,
at the end of a given round $r_1$, the adversary starts devoting their mining
power to block suppression starting from the next round.

The block suppression attack works as follows. Let $b$ be the honestly generated
$\mu$-superblock which was diffused at the end of the previous round. If the
round was not uniquely successful, let $b$ be any of the diffused
honestly-generated $\mu$-superblocks. Let $b$ be the tip of an honest chain
$\chain_B$. The adversary first mines on top of $\chain_B[-2]$. If she is
successful in mining a block $b'$, she continues extending the chain ending
at $b'$ (to mine $b''$ and so on). The value $r$ is fixed, so the adversary
devotes exactly $r$ rounds to this whole process; the adversary will keep mining
on top of $\chain_B[-2]$ (or one of the adversarially-generated extensions of
it) for exactly $r$ rounds, regardless of whether $b'$ or $b''$ have been found.
At the same time, the honest parties will be mining on top of $b$ (or a
competing block in the case of a non-uniquely successful round). Again, further
successful block diffusion by the honest parties shall not affect that the
adversary is going to spend exactly $r$ rounds for suppression.
This attack will succeed with overwhelming probability for the right choice
of protocol values.

\begin{theorem}[Double-spending attack]
There exist parameters $p, n, t, q,  \mu, \delta$, with $t\leq (1-\delta)(n-t)$,
and a double spending attack against KLS PoPoW that succeeds with overwhelming
probability.
\end{theorem}
\import{./}{proofs/attack.tex}

\noindent\textbf{Remark.}
It is worth isolating the mistake in the security proof from
the interactive construction paper \cite{KLS}. Suppose player $B$ is honest and
player $\mathcal{A}$ is adversarial and suppose $b$, the LCA block, was honestly
generated and suppose that the superchain comparison happens at level $\mu$.
Their security proof then correctly argues that there will have been more
honestly- than adversarially-generated $\mu$-superblocks after block $b$.
Nevertheless, we observe that the mere fact that there have been more honestly-
than adversarially-generated $\mu$-superblocks after $b$ does not imply that
$|\overline\pi_\mathcal{A}\upchain^\mu\{b:\}| \leq
|\overline\pi_B\upchain^\mu\{b:\}|$. The reason is that some of these
superblocks could belong to blocktree forks that have been abandoned by $B$.
Thus, the security conclusion does not follow. Regardless, their basic argument
and construction is what we will use as a basis for constructing a system that
is both provably secure and succinct under the same assumptions, albeit
requiring a more complicated construction structure to obtain security.
