\subsection{Two-way-pegged tokens.}
\label{sec:sidechains}
\anote{Rewrite this entirely based on the cross-chain scenario.}
It is widely known that Bitcoin faces significant scaling
hurdles~\cite{onscaling}, but upgrading Bitcoin is notoriously difficult. A
widely anticipated solution is to treat the Bitcoin blockchain as a host for
``sidechains,'' which are proof-of-work blockchains separate from Bitcoin, but
that can be backed by Bitcoin deposits. Our efficient NIPoPoW construction
solves an open problem posed by Back et al., and thus enables this vision.

\anote{Include a feature comparison, with Rootstock/Federated, Drivechain and BTCRelay}

\begin{table}[t]
  \begin{tabular}{r|rrr}
              & Decentralized & Cost        & \\
    \hline
    Federated\cite{rootstock,federated}
              &  & $O(1)$ & \\
    Drivechain,BTCRelay\cite{drivechain,btcrelay,peacerelay}
              & x& $O(N)$ & \\
    Ours\cite{nipopow}
              & x& $O(polylog N)$ & \\
  \end{tabular}
\end{table}

The idea behind sidechains is to implement an SPV verifier for one blockchain
(the ``sidechain'') as a smart contract within another blockchain (the ``host
chain''). An (inefficient) implementation of this idea, called
BTCRelay~\cite{ethereum}, is already running on Ethereum. It comprises a
smart contract that allows users to submit Bitcoin block headers, which it validates and
stores. This allows Ethereum smart contracts to condition their behavior based
on committed Bitcoin transactions. The downside is that every Bitcoin header must be
stored, which is expensive due to the cost of Ethereum blockchain storage (7 cents per kB). If
Bitcoin were upgraded with the interlink data structure, the BTCRelay
functionality could be provided inexpensively.

The most anticipated use of sidechains is to implement a virtual asset on the
sidechain backed by currency on the host chain (a two-way-peg)~\cite{sidechains}. This is accomplished by defining two new transaction
types: ``deposit'' transactions lock coins on the host chain and create new
assets on the side chain; ``withdrawal'' transactions do the opposite. Deposit
transactions committed on the host chain are delivered to the sidechain using a
proof; vice versa for withdrawal transactions. Whereas in the multi-blockchain
wallet application the client queries each of its peers (one of which is assumed
to be honest), in this setting peers must listen for event notifications when a
``withdrawal'' transaction is commited, and if they have a better NIPoPoW proof
that invalidates this claim, they must submit it as a transaction within some
time bound. The smart contract should collect a security deposit from the
withdrawer that can be used to compensate peers for providing invalidation
claims. Succinctness (and certificates of badness) are thus especially important
for sidechains, since the size of the largest proof under normal conditions
should determine what is an acceptable security deposit.
